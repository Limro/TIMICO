\documentclass[compressed, presentation, notheorems, 12pt]{beamer}

\input{Preamble}


\title{Group 3's DDS system}
\subtitle{Slides and presentation}
\author[Author, Anders] % (optional, for multiple authors)
{L.~Brøsted \and R.~Bækgaard \and M.~Manø \and C.~Werge}
\institute
{
  Aarhus University \\
  School of Engineering
}
\date{TIMICO, 2014}
\subject{DDS}

\begin{document}
 	\frame{\titlepage}

\begin{frame}
\frametitle{Table of Contents}
\tableofcontents%[currentsection]
\end{frame}

	
\section{The slave}
	\begin{frame}{The slave's role}	

	\begin{itemize}
		\item Receive timestamps from the master
		\item Make delay request
	\end{itemize}
	
	\end{frame}


\subsection{IDL generation}
	\begin{frame}[containsverbatim]{The IDL file}

	\begin{lstlisting}[style=Code-C++]
module PTP
{
  enum MsgType
  {
    SYNC,
    FOLLOWUP,
    DELAYREQUEST,
    DELAYRESPONSE
  };
  struct Msg
  {
    MsgType type;         
    unsigned long long value;     
  };
};
	\end{lstlisting}
	\end{frame}



\subsection{Slave functions}

	\begin{frame}[containsverbatim]{Main loop}
	
	\begin{lstlisting}[style=Code-C++]
using(var slave = new Slave())
{
	slave.setup(domain_id: 0);
	
	const int receive_period = 2000; 
	while (keepRunning)
	{
		long delayTs, delayTm;
		slave.SendRequest(out delayTs, out delayTm);
		slave.setDelay(delayTs, delayTm);

		Thread.Sleep(receive_period);
	}
}
	\end{lstlisting}

	\end{frame}




	\begin{frame}[containsverbatim]{Handle the data received}
	\begin{lstlisting}[style=Code-C++]
void handleDataReceived(Msg data)
{
   switch (data.type)
	{
		case SYNC:
			syncTs = clock.Now.Ticks;
			break;
		case FOLLOWUP:
			clock.Offset = syncTs - data.value - Delay;
			syncTm = data.value;
			break;
	}
}
	\end{lstlisting}
	\end{frame}




	\begin{frame}[containsverbatim]{Setup, part 1/2}
	\begin{lstlisting}[style=Code-C++]
void setup(int domain_id)
{
	participant= createParticipant(domain_id);
	subscriber= createSubscriber(participant);		
	var type_name = MsgTypeSupport.get_type_name();
	MsgTypeSupport.register_type(participant, type_name);

	\end{lstlisting}
	\end{frame}


	\begin{frame}[containsverbatim]{Setup, part 2/2}
	\begin{lstlisting}[style=Code-C++]

	topic = createTopic(participant, type_name, "PTP");
	reader = createReader(participant, subscriber, topic, reader_listener);

	requester = new Requester<Msg, Msg>(participant, "PTP",
		MsgTypeSupport.get_instance(),
		MsgTypeSupport.get_instance());

}
	\end{lstlisting}
	\end{frame}



	\begin{frame}[containsverbatim]{SendData 1/2}
	\begin{lstlisting}[style=Code-C++]
void SendRequest(out long ts, out long tm)
{
	var request = new Msg { type = DELAYREQUEST };
	ts = clock.Now.Ticks;
	requester.SendRequest(request);

	var reply = requester.CreateReplySample();
	var received = requester.ReceiveReply(reply, maxWait);

	\end{lstlisting}
	\end{frame}



	\begin{frame}[containsverbatim]{SendData 2/2}
	\begin{lstlisting}[style=Code-C++]
	if (received)
	{
		if (reply.Info.valid_data)
		{
			if(reply.Data.type == DELAYRESPONSE)
			{
				tm = reply.Data.value;
			}
		}
	}
}
	\end{lstlisting}
	\end{frame}




	\begin{frame}[containsverbatim]{Retrieving the time}
	\begin{lstlisting}[style=Code-C++]
long GetTime()
{
	var elapsed = sw.Elapsed.Ticks;
	now += elapsed - lastElapsed;

	lastElapsed = elapsed;

	return now;
}
	\end{lstlisting}
	\end{frame}



	\begin{frame}[containsverbatim]{Respons from slave}
	\begin{lstlisting}[style=Code-C++]
class MsgReplierListener : SimpleReplierListener<Msg, Msg>
{
	Msg reply = new Msg {type=DELAYRESPONSE};

	Msg OnRequestAvailable(Sample<Msg> req)
	{
		reply.type = DELAYRESPONSE;
		reply.value = Master.getCurrentTime();
		return reply;
	}
}
	\end{lstlisting}

	\end{frame}




\section{Master}

	\begin{frame}{The master}
	\begin{itemize}
		\item Must send its time to the slave.
		\item Must answer the slave when asking for time.
	\end{itemize}
	\end{frame}


\subsection{Master functions}
	\begin{frame}[containsverbatim]{Retrieving the time}
	\begin{lstlisting}[style=Code-C++]
using (var master = new Master())
{
	master.setup(domain_id: 0);
	const int send_period = 485; 

	while (keepRunning)
	{
		var tm = getCurrentTime();

		master.sendSync(tm + jitter());
		master.sendFollowup(tm);						

		Thread.Sleep(send_period);
	}
}
	\end{lstlisting}
	\end{frame}

 	

	\begin{frame}[containsverbatim]{Sending data to the slave}
	\begin{lstlisting}[style=Code-C++]
public void sendSync(long estimatedTm)
{
	instance.type = MsgType.SYNC;
	instance.value = estimatedTm;
	writer.write(instance, ref instance_handle);
}
	\end{lstlisting}

	The function \code{sendFollowup(long tm)} is very similar.
	\end{frame}





\end{document}
